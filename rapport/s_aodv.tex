\chapter{AODV}
    Ad-hoc On-demand Distance Vector (AODV) est un protocole réactif à vecteur de distance.\\
    Ce protocole définit 3 types de messages:
    \begin{enumerate}
        \item Route request (\textit{RREQs})
        \item Route Remplies (\textit{RREPs})
        \item Route Errors (\textit{RERRs})
    \end{enumerate}

%%%%%%%%%%%%%%%%%%%%%%%%%%%%%%%%%%%%%%%%%%%%%%%%%%%%%%%%%%%%%%%%%%%%%%%%%%%%%%%
    \underline{\textbf{Format des paquets}}\\
        \textbf{RREQ}
        \begin{figure}[H]
        \centering
            \begin{verbatimtab}
 0                   1                   2                   3
 0 1 2 3 4 5 6 7 8 9 0 1 2 3 4 5 6 7 8 9 0 1 2 3 4 5 6 7 8 9 0 1
+-+-+-+-+-+-+-+-+-+-+-+-+-+-+-+-+-+-+-+-+-+-+-+-+-+-+-+-+-+-+-+-+
|     Type      |J|R|G|D|U|   Reserved          |   Hop Count   |
+-+-+-+-+-+-+-+-+-+-+-+-+-+-+-+-+-+-+-+-+-+-+-+-+-+-+-+-+-+-+-+-+
|                            RREQ ID                            |
+-+-+-+-+-+-+-+-+-+-+-+-+-+-+-+-+-+-+-+-+-+-+-+-+-+-+-+-+-+-+-+-+
|                    Destination IP Address                     |
+-+-+-+-+-+-+-+-+-+-+-+-+-+-+-+-+-+-+-+-+-+-+-+-+-+-+-+-+-+-+-+-+
|                  Destination Sequence Number                  |
+-+-+-+-+-+-+-+-+-+-+-+-+-+-+-+-+-+-+-+-+-+-+-+-+-+-+-+-+-+-+-+-+
|                    Originator IP Address                      |
+-+-+-+-+-+-+-+-+-+-+-+-+-+-+-+-+-+-+-+-+-+-+-+-+-+-+-+-+-+-+-+-+
|                  Originator Sequence Number                   |
+-+-+-+-+-+-+-+-+-+-+-+-+-+-+-+-+-+-+-+-+-+-+-+-+-+-+-+-+-+-+-+-+
            \end{verbatimtab}
        \caption{format d'un paquet RREQ \cite{aodv_w}}
        \label{rreqPaquet}
    \end{figure}
    Où J,R,G,D,U sont des flags.
        \newpage
        \textbf{RREP}
        \begin{figure}[H]
            \centering
                \begin{verbatimtab}
 0                   1                   2                   3
 0 1 2 3 4 5 6 7 8 9 0 1 2 3 4 5 6 7 8 9 0 1 2 3 4 5 6 7 8 9 0 1
+-+-+-+-+-+-+-+-+-+-+-+-+-+-+-+-+-+-+-+-+-+-+-+-+-+-+-+-+-+-+-+-+
|     Type      |R|A|    Reserved     |Prefix Sz|   Hop Count   |
+-+-+-+-+-+-+-+-+-+-+-+-+-+-+-+-+-+-+-+-+-+-+-+-+-+-+-+-+-+-+-+-+
|                     Destination IP address                    |
+-+-+-+-+-+-+-+-+-+-+-+-+-+-+-+-+-+-+-+-+-+-+-+-+-+-+-+-+-+-+-+-+
|                  Destination Sequence Number                  |
+-+-+-+-+-+-+-+-+-+-+-+-+-+-+-+-+-+-+-+-+-+-+-+-+-+-+-+-+-+-+-+-+
|                    Originator IP address                      |
+-+-+-+-+-+-+-+-+-+-+-+-+-+-+-+-+-+-+-+-+-+-+-+-+-+-+-+-+-+-+-+-+
|                           Lifetime                            |
+-+-+-+-+-+-+-+-+-+-+-+-+-+-+-+-+-+-+-+-+-+-+-+-+-+-+-+-+-+-+-+-+
                \end{verbatimtab}
            \caption{format d'un paquet RREP \cite{aodv_w}}
            \label{rreqPaquet}
        \end{figure}
        Où R et A sont des flags.

%%%%%%%%%%%%%%%%%%%%%%%%%%%%%%%%%%%%%%%%%%%%%%%%%%%%%%%%%%%%%%%%%%%%%%%%%%%%%%%
    \underline{\textbf{Découverte d'un chemin}}\\
        La découverte d'un chemin est intiée par un noeud voulant envoyer des paquets à une destination pour laquelle il n'a aucune information.\\
        Chaque noeud possède deux compteurs: \textbf{\textit{sequence\_number}} et \textbf{\textit{rreq\_id}}.\\
        
        \textbf{Génération du \textit{RREQ}}\\
            Le noeud source incrémente ses compteurs \textbf{\textit{sequence\_number}} et \textbf{\textit{rreq\_id}} de 1.
            Il envoie ensuite un \textit{RREQ} en broadcast à ses voisins.\\
        
        \textbf{Propagation du \textit{RREQ}}\\
        \begin{itemize}
            \item[$\bullet$] \underline{Noeud intermédiaire}\\
                A la réception d'un \textit{RREQ}, un noeud intermédiaire va pouvoir rajouter ou mettre à jour
                les routes vers le noeud source du \textit{RREQ} et vers son prédécesseur.\\
                Ensuite deux situations sont possibles:
                \begin{enumerate}
                    \item Le noeud courant possède une route active vers la destination et le numéro de séquence de la route est plus grand 
                        ou égal au numéro de séquence de la destination dans le \textit{RREQ}.\\
                        Dans ce cas, il peut envoyer par unicast un \textit{RREP} à la source du \textit{RREQ}
                    \item La condition en 1. n'est pas satisfaite.\\
                        Dans ce cas, le noeud va incrémenter le nombre de sauts du \textit{RREQ} et le propager à ses voisins.\\
                \end{enumerate}
                

            \item[$\bullet$] \underline{Noeud destination}\\
                A la réception d'un \textit{RREQ} lui étant destiné, un noeud va, comme un noeud intermédiaire, 
                rajouter ou mettre à jour les routes vers le noeud source du \textit{RREQ} et vers son prédécesseur.\\
                Si le \textbf{\textit{Destination Sequence Number}} du \textit{RREQ} est égale à son \textbf{\textit{sequence\_number}},
                il va incrémenter ce dernier et envoyer un \textit{RREP} en unicast vers la source du \textit{RREQ}.\\
             
        \end{itemize}


        \textbf{Propagation du \textit{RREP}}\\
            A la réception d'un \textit{RREP}, un noeud va pouvoir rajouter ou mettre à jour
            les routes vers le noeud source du \textit{RREP} et vers son successeur.\\
            Il va ensuite incrémenter le nombre de sauts du \textit{RREP} et le propager en unicast vers la destination de ce \textit{RREP}.\\

%%%%%%%%%%%%%%%%%%%%%%%%%%%%%%%%%%%%%%%%%%%%%%%%%%%%%%%%%%%%%%%%%%%%%%%%%%%%%%%
    \underline{\textbf{Table de routage}}\\
        Chaque entrée d'une table de routage contient les informations suivantes:
        
        \begin{table}[H]
            \centering
            \begin{tabular}{|l|l|}
                \hline
                \textit{dest}       & Adresse IP de destination\\
                \textit{dest\_SN}   & Numéro de séquence de destination\\
                \textit{flag}       & Indicateur de numéro de séquence de destination valide\\
                \textit{out}        & Interface réseau\\
                \textit{hops}       & Comptage de sauts (nombre de sauts nécessaires pour atteindre la destination)\\
                \textit{next-hop}   & Prochain saut\\
                \textit{precursors} & Liste des précurseurs\\
                \textit{lifetime}   & temps d'expiration ou de suppression de l'itinéraire\\
                \hline
            \end{tabular}
            \caption{entrée d'une table de routage \aodv \cite{aodv_w}}
            \label{routingTable_aodv}
        \end{table}

        \textbf{Mise à jour de la table de routage}\\
            Soit N une nouvelle route et O la route existante.\\
            O est mise à jour si:\\
            \begin{center}
                \begin{tabular}{|l|}
                    \hline
                    $O.SN \leq N.SN$ \\
                    \textbf{ou} ($O.SN = N.SN$ \textbf{et} $O.hop\_count > N.hop\_count$)\\
                    \hline
                \end{tabular}
            \end{center}
        
        \textbf{Gestion du \textit{lifetime}}\\
            Le temps de vie d'une route dans la table de routage est initialisé à \textit{active\_route\_timeout} (3 millisecondes).\\
            Quand ce timer expire, la route passe de active à inactive. Une route inactive ne pourra plus être utilisée pour transférer des données
            mais pourra fournir des informations pour de futurs \textit{RREQ} et la réparation de routes.\\
            Quand une route est utilisée, son temps de vie  est actualisé à: $current time + active\_route\_timeout$

%%%%%%%%%%%%%%%%%%%%%%%%%%%%%%%%%%%%%%%%%%%%%%%%%%%%%%%%%%%%%%%%%%%%%%%%%%%%%%%
    \vspace{1cm}
    \underline{\textbf{Evitement de boucles}}\\
        A priori les numéros de séquences suffisent pour éviter les boucles. Cependant, d'après \cite{loop_aodv_w}, il y a des
        ambiguités dans le RFC \cite{aodv_w}. Dû à ces ambiguités, l'implémentation pourrait introduire des boucles.
        %Certaines parties du RFC concerant la gestion des numéros de séquences pourraient introduiredes boucles. 
        Nous approfondirons la lecture de cet article si nous choisissons ce protocole afin d'éviter les boucles dans notre implémentation.

%%%%%%%%%%%%%%%%%%%%%%%%%%%%%%%%%%%%%%%%%%%%%%%%%%%%%%%%%%%%%%%%%%%%%%%%%%%%%%%
    \underline{\textbf{Défaillance d'un lien}}\\
        Un noeud faisant partie d'une route active broadcast des messages \textit{hello} (RREP)
        régulièrement.\\
        Si un noeud ne reçoit pas de message durant un certain temps pour un voisin, il va considérer
        que le lien avec ce voisin est perdu.\\
        Dans ce cas, il va en informé ses voisins impactés par un RERR.