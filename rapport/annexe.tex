\vspace{0.5cm}
\textbf{Multicasting}\\
    Le multicasting permet d'envoyer simultanément un paquet \espmesh\ à plusieurs noeuds du réseau. Le multicasting
    peut être réalisé en spécifiant
    \begin{itemize}
        \item Soit un ensemble d'adresses \mac\\
            Dans ce cas, l'adresse de destination doit être
            {\fontfamily{qcr}\selectfont \textls[-300]{\small 0 1 : 0 0 : 5 E : x x : x x : x x}}
            Cela signifie que le paquet est un paquet multicast et que la liste des adresses peut être obtenue dans les options du header.
        \item Soit un groupe préconfiguré de noeuds\\
            Dans ce cas, l'adresse de destination du paquet doit être l'ID\footnote{Dans un réseau \espmesh, chaque groupe a un ID unique
            ayant la même structure qu'une adresse mac (par exmple {\small 7 7 : 7 7 : 7 7 : 7 7 : 7 7 : 7 7})}
            du groupe et un flag \textsc{mesh\_data\_group} doit être ajouté. % todo ajouté où ?
            % todo comment les noeuds sont au courant qu'ils sont dans ce groupe
    \end{itemize}

\vspace{0.5cm}
\textbf{Broadcasting}\\
    Le broadcasting permet de transmettre un paquet \espmesh\ à tous les noeuds du réseau. Pour éviter de gaspiller de
    la bande passante, \espmesh\ utilise les règles suivantes:
    \begin{enumerate}
        \item Quand un noeud intermédiare reçoit un paquet broadcast de son parent, il va le transmettre à tous ses enfants
            et en stocker une copie
        \item Quand un noeud intermédiaire est la source d'un paquet broadcast, il va le transmettre à son parent et à ses enfants
        \item Quand un noeud intermédiaire reçoit un paquet d'un de ses enfants, il va le transmettre à ses autres enfants, son parent
            et en stocker une copie
        \item Quand une feuille est la source d'un paquet broadcast, elle va le transmettre à son parent
        \item Quand la racine est la source d'un paquet broadcast, elle va le transmettre à ses enfants
        \item Quand la racine reçoit un paquet broadcast de l'un de ses enfants, elle va le transmettre à ses autres enfants et en stocker une copie
        \item Quand un noeud reçoit un paquet broadcast avec son adresse \mac\ comme adresse source, il l'ignore
        \item Quand un noeud intermédiaire reçoit un paquet broadcast de son parent, s'il possède une copie de ce paquet (càd que ce paquet a été à l'origine transmis par l'un de ses enfants), il va l'ignorer
            pour éviter les cycles ( protocole d'inondation)
    \end{enumerate}