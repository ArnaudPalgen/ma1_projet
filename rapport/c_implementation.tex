\chapter{Implémentation}

%%%%%%%%%%%%%%%%%%%%%%%%%%%%%%%%%%%%%%%%%%%%%%%%%%%%%%%%%%%%%%%%%%%%%%%%%%%%%%%
\section{Limitations}
        Le driver Wi-Fi d'\textit{IDF} ne nous permet pas d'avoir une connexion
        avec plusieurs noeuds simultanément. \espnow\ pourraît être une solution
        pour palier à ce problème.\\
    
    
        \textbf{ESP NOW}\\
            \espnow\ est un protocole de communication Wi-Fi sans connexion défini
            par Espressif.\\ Nous n'avons trouvé aucune documentation décrivant le
            fonctionnement de ce protocole.\\
            Avec la documentation disponible, nous savons qu'un noeud a une liste
            de \textit{peers} (ses voisins) avec qui il peut échanger des données.
            Le nombre de voisins est limité à 20. Cette limite ne posera pas 
            problème pour ce projet.
        
%%%%%%%%%%%%%%%%%%%%%%%%%%%%%%%%%%%%%%%%%%%%%%%%%%%%%%%%%%%%%%%%%%%%%%%%%%%%%%%
\section{Prochaines étapes}
        \begin{enumerate}
            \item Nous allons créer un réseau \espmesh. Ceci nous permettra de nous
            familiariser avec l'environnement \textit{IDF}.
            \item Nous évaluerons les performances et fonctionnalités de ce protocole.
            \item Nous implémenterons un protocole de routage mesh choisis plus haut.
            \item L'objectif est de l'implémenter au niveau de la couche liaison de données.
                Si nous rencontrons trop de difficulté ou que nous jugeons que ce 
                choix n'est pas judicieux, nous implémenterons ce protocole au niveau 
                de la couche réseau.
            \item Nous étudierons les différentes possibilités d'économies d'énergie
            \item Nous évaluerons les performances et fonctionnalités du prorotype créé. 
            
        \end{enumerate}